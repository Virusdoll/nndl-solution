\documentclass[../main.tex]{subfiles}

\begin{document}

两类线性可分:
\begin{align*}
yf((\rho x_1 + (1-\rho) x_2), w)
= & y\big( w^T (\rho x_1 + (1-\rho) x_2) + b \big) \\
= & y \big( w^T \rho x_1 + w^T (1-\rho) x_2 + b \big) \\
= & y \big( w^T \rho x_1 + \rho b \big)
    + y \big( w^T (1-\rho) x_2 + (1-\rho) b \big) \\
= & \rho y \big( w^T x_1 + b \big)
    + (1-\rho) y \big( w^T x_2 + b \big) \\
> & 0
\end{align*}
即,
$yf((\rho x_1 + (1-\rho) x_2), w) > 0$
\\

多类线性可分:
\begin{align*}
f_c(\rho x_1 + (1-\rho) x_2)
= & w_c^T\big( \rho x_1 + (1 - \rho) x_2 \big) + b_c \\
= & \rho \big( w_c^T x_1 + b_c \big)
    + (1-\rho) \big( w_c^T x_2 + b_c \big) \\
> & \rho \big( w^T_{\tilde{c}} x_1 + b_{\tilde{c}} \big)
    + (1-\rho) \big( w^T_{\tilde{c}} x_2 + b_{\tilde{c}} \big) \\
= & w^T_{\tilde{c}}\big( \rho x_1 + (1 - \rho) x_2 \big) + b_{\tilde{c}} \\
= & f_{\tilde{c}}(( \rho x_1 + (1 - \rho) x_2), w_{\tilde{c}})
\end{align*}
即,
$
f_{\tilde{c}}(\rho x_1 + (1-\rho) x_2)
> f_{\tilde{c}}(( \rho x_1 + (1 - \rho) x_2), w_{\tilde{c}})
$

\end{document}